\documentclass[a4paper,11pt]{article}
\usepackage[top=2cm,bottom=2cm,left=2cm,right=2cm,asymmetric]{geometry}
\usepackage{url}
\usepackage{paralist}
\usepackage{authblk}
\usepackage[pdftex,colorlinks=true,hyperfootnotes=false]{hyperref}

\title{\vspace{-4em}The Problem of the P3:\\Public-Private Partnerships in National Cyber Security Strategies}

\author[1]{Madeline Carr}
\author[2]{Tom Crick}
\affil[1]{Department of International Politics, Aberystwyth University}
\affil[2]{Department of Computing \& Information Systems, Cardiff Metropolitan University}
\affil[1]{\protect\url{madeline.carr@aber.ac.uk}}
\affil[2]{\protect\url{tcrick@cardiffmet.ac.uk}}


\renewcommand\Authands{ and }
\def\UrlBreaks{\do\/\do-}

\date{ }

\begin{document}
\maketitle

\begin{abstract}
Cyber security is an emerging -- and high profile -- national security
problem; not only in terms of material vulnerabilities but also in
terms of conceptualising security approaches. Many states
(particularly liberal Western democracies) have situated the
'public-private partnership' (P3) at the centre of their national
cyber security strategies. However, there has been a persistent
ambiguity around this fundamental concept. Policymakers regard the
state as without the capability and also without the mandate to impose
security requirements beyond government-owned systems. The private
sector, however, is highly averse to accepting responsibility for
national security and will fund cyber security only within the
parameters of the profit/risk calculation appropriate for a
shareholder-based arrangement. Amidst increasing suggestions that a
market-led approach to cyber security has failed, a deeper look at the
ideas and concepts behind this approach finds that a reliance on the
P3 emerges from deeply held and shared beliefs about government
legitimacy and private authority which may not be easily reconciled
with wider national security issues in a modern digital economy.
\end{abstract}

\section{Introduction}

Cyber security is emerging as one of the most challenging aspects of
the information age for policymakers and scholars of international
relations. It has implications for national security, the economy,
human rights, civil liberties and international legal
frameworks. Although politicians have been aware of the threats of
cyber insecurity since the early years of Internet
technology~\cite{clinton:1992}, anxiety about the difficulties in
resolving or addressing them has increased rather than
abated~\cite{obama:2009}. In response, governments have begun to
develop national cyber security strategies to outline the way in which
they intend to address cyber insecurity. In many states where critical
infrastructure such as utilities, financial systems and transport has
been privatised, these policy documents are heavily reliant upon what
is referred to as the `public-private partnership' as a key mechanism
through which to mitigate the threat. In the US and the UK, the
public-private partnership has repeatedly been referred to as the
`cornerstone' or `hub' of cyber security
strategy~\cite{clinton:1992,gwbush:2003,maude:2012}.

While public-private partnerships have often been developed as an
appropriate means to address both non-traditional and traditional
security threats~\cite{manwaring:2002,usdoc:2012}, in the context of
cyber security this arrangement is uniquely problematic. There has
been a persistent ambiguity with regard to any clear and agreed
parameters for the partnership. The reticence of politicians to claim
authority for the state to legislate tougher cyber security measures
coupled with the private sector’s aversion to accepting responsibility
or liability for national security leaves the `partnership' without
clear lines of responsibility or accountability. Questions are now
being raised (including by Obama and the US Government) about the
efficacy of a market-driven approach to cyber security, although any
alternative in liberal democratic states has yet to
emerge~\cite{obama:2009}. Crucially, questions arise here about the
extent to which the state can be seen to be abdicating not just
authority but responsibility for national security. As Dunn Cavelty
and Suter~\cite{dunncavelty+suter:2009} point out in their article on
this topic, `generating security for citizens is a core task of the
state; therefore it is an extremely delicate matter for the government
to pass on its responsibility in this area to the private sector'.
Essentially, this raises questions about how well the state is
equipped to provide national security in this context and about how
existing policies and practices of national security are being
challenged by this new threat conception.

This paper develops a comprehensive understanding of how policymakers
and the private sector are conceptualising their respective roles in
national cyber security, where there may be disparity in these
conceptions and what implications this may have for national and
international cyber security. The paper moves onto the analysis of the
public-private partnership from the perspectives of both partners. It
should be noted here that there is a round of interviews yet to be
completed for this project which will contribute further to the
analysis. What is presented here is the conceptual framework and the
outcome of documentary research.

\section{Analysis of the Public-Private Partnership in Cyber Security}

There are several reasons why cyber security, particularly in the
context of critical infrastructure protection, has been conceived of
as some kind of collaborative project for the public and private
sectors. The state is understood to be responsible for the provision
of security, especially national security. Critical infrastructure,
those assets and systems necessary for the preservation of national
security (broadly defined), is perceived as an integral part of
providing security to the state. The potential implications of a large
scale cyber attack on critical infrastructure are so extensive that it
follows naturally that the government would recognise some authority
and responsibility here. However, because most of the critical
infrastructure in the US and UK is privately owned and operated, by
definition there has to be some kind of relationship between the
public and private sector in terms of the provision of security in
this context. 

The public-private partnership is not of course, unique to cyber
security. It has been employed widely by states like the US and UK as
a mechanism to deal with a range of other issues including security
related ones. The practice intensified from the 1990s when the
privatisation of critical infrastructure was regarded as economically
beneficial to the state, freeing up capital and relying more heavily
on the efficiencies and business practices of the private
sector. There is an extensive body of literature that has developed in
the wake of this shift that examines the public-private partnership in
all kinds of contexts. It deals with the background of these
partnerships, the range of different approaches, how to measure
success and failure, and how responsibility and authority are
delegated. There has also been some examination of the public-private
partnership in cyber security, most notably by Dunn Cavelty and
Suter~\cite{dunncavelty+suter:2009}, but this focuses on ways to
improve it rather than critically analysing the political implications
of it. Combined, this literature provides a solid foundation in
highlighting the ways in which this partnership is distinct but also
by outlining common assumptions and expectations that run through
public-private partnerships more generally.

\subsection{What is this public-private partnership?}

It is necessary to be clear about what exactly is meant by the
term public-private partnership in this particular context. Perhaps
not unexpectedly, there is a huge range of diverse arrangements that
are referred to as public-private partnerships, ranging from the joint
provision of services with some government regulatory oversight
(health sectors), to closely contracted outsourcing of large
infrastructure projects, (building roads and bridges, the Olympics,
etc). Much of the literature on public-private partnerships revolves
around identifying and classifying partnership arrangements. This
often takes place within a framework of authority and responsibility
–- key concepts for this study. In examining these relationships,
Wettenhall~\cite{wettenhall:2003} identifies two broad categories: a)
horizontal, non-hierarchical arrangements characterised by consensual
decision-making and b) hierarchically organised relationships with one
party in a controlling role. The implication being, he argues, that
true `partnerships' are of type a) and not type b).

This distinction has implications for the public-private partnership
in cyber security. National cyber security strategies avoid
suggestions of hierarchy when they refer to the public/private
partnership. The language is deliberately cooperative and implies a
shared purpose and shared interests. The UK Cyber Security
Strategy~\cite{ukcss:2011} states that achieving the goal of a safe,
secure Internet will `require everybody, the private sector,
individuals and government to work together. Just as we all benefit
from the use of cyberspace, so we all have a responsibility to help
protect it.' With specific reference to the role of the private
sector, it states that there is an expectation that the private sector
will `work in partnerships with each other; Government and law
enforcement agencies, sharing information and resources, to transform
the response to a common challenge, and actively deter the threats we
face in cyberspace'~\cite{ukcss:2013}.  This non-hierarchical language
belies the poor alignment of perceptions about the `common challenge'
and the `threats we face in cyberspace'~\cite{uknao:2013}. It assumes
that those are the same for the public and private sector when in
fact, they are not. The private sector regards cyber security
challenges as financial and reputational -- not as a common public
good which is how governments regard national cyber security.

On a more granular level, Linder~\cite{linder:1999} identifies six
distinctive uses of the term P3 and links them to neo-liberal or
neo-conservative ideological perspectives. In doing so, he draws out
questions about their intended purpose and significance as well as
`what the relevant problems are to be solved and how best to solve
them.'  Two of these `types' can shed light on what is meant by the
public-private partnership in cyber security; {\emph{partnership as
management reform}} and {\emph{partnership as power sharing}}.

Linder argues that partnership as management reform refers to the
expectation that government managers will learn `by emulating their
partners' and shift their focus from administrative processes to
deal-making and attracting capital in a more entrepreneurial and
flexible approach. Significantly, this is regarded as one of the
objectives of the partnership because of the belief that the market is
inherently superior and `its competitive character stimulates
innovation and creative problem solving' -- a view embedded in
neo-liberalism~\cite{linder:1999}. Perhaps not surprisingly, although
this is reflected in the strategies of both states, it is much more
pronounced in the US documents.

The [George W.] Bush Administration’s National Strategy to Secure
Cyberspace~\cite{gwbush:2003} argued that in the US ``traditions of
federalism and limited government require that organizations outside
the federal government take the lead'' in cyber security.  This
interpretation of the government's limited authority is combined here
with an assumption of its limited capability. ``The federal government
could not -- and, indeed, should not -- secure the computer networks
of privately owned banks, energy companies, transportation firms, and
other parts of the private sector.''  This is based on the belief that
``in general, the private sector is best equipped and structured to
respond to an evolving cyber threat'' and, at a US Congressional
hearing in 2000, Deputy Attorney General Eric Holder’s statement that
decision makers in the US ``believe strongly that the private sector
should take the lead in protecting private computer
networks.''~\cite{holder:2000} In testimony before a hearing on
internet security, the FBI's Michael Vatis argued that cyber security
is ``clearly the role of the private sector. The Government has
neither the responsibility nor the expertise to act as the private
sector’s system administration.''~\cite{vatis:2000}.

So there is a rejection here of government liability for private
networks that is framed in the belief that the government has neither
the authority nor the capability to deal with cyber security. It is an
approach in keeping with the partnership as management reform type
identified by Linder -- though the government rejects the objective of
change inherent within that type. Rather, it promotes two `truths'
about the private sector. First, they must take responsibility and
liability for their own network security and second, their superior
capacity for flexibility and innovation means that they are best
placed take the lead on this particular security problem. The problem
of course, is that these networks are central to national security and
therein lies the problem from the perspective of the private sector.

The private sector develops security strategy within a very different
framework to that of the government's `public good' conception. For
the private operators of critical infrastructure, decisions are made
within a business model that responds to profit margins and
shareholder interests. This is largely incompatible with the promotion
of a `public good'. The private sector raises two main objections to
the role that the government perceives for them in the cyber security
strategies. First, they argue that the expense of ensuring cyber
security to a national security level would be significant and second,
that the litigious nature of (especially US) society means that
industry would be very resistant to accepting liability for the
security of their products or systems~\cite{paller:2005}.  

Stiglitz and Wallsten~\cite{stiglitz+wallsten:1999} make some
important observations about this dichotomised approach to
public-private partnerships in the context of technology
innovation. `Theory predicts' they argue, `and many empirical studies
confirm, that profit-maximising firms invest less than the socially
optimal level of [technology research and development].' What is in
society’s best interest with regard to cyber security, is not always
in the best interests of the private sector. This is because, they
argue, social benefits do not translate in terms of private
profitability -- no matter how desirable the outcome.

So private sector owners of critical infrastructure accept
responsibility for securing their systems -- to that point that it is
profitable. That is, that the cost of dealing with an outage promises
to cost more than prevention. However, they tend to make a distinction
between protecting against the low-level threat such as ``background
noise, individual hackers, and possibly hacktivists'' and protecting
against an attack on the state (national security). In testimony at a
US hearing on privately owned critical infrastructure cyber security,
one witness explained that ``it is industry’s contention that
government should protect against the larger threats -- organized
crime, terrorists, and nation-state threats -- either through
law-enforcement or national defense.''~\cite{vanardo:2005}.

This disjuncture in perceptions is arguably at the heart of the
tension in this `partnership'. Typically, the rationale articulated in
the literature for partnering is that neither partner on its own can
achieve their desired objectives. They must either need each other or
there must be a financial arrangement that makes the partnership
attractive. This, we can observe most readily in the single most
emphasised practice in this partnership -- information sharing. And
information sharing can be understood in the second of Linder's
`types' of public-private partnerships -- partnerships as power
sharing.

Linder writes that partnerships as power sharing are based on an ethos
of cooperation where `trust replaces the adversarial relations endemic
to command-and-control regulation' and in which there is some mutually
beneficial sharing of responsibility, knowledge, or risk. In most
instances, he writes, `each party brings something of value to the
others to be invested or exchanges'. Finally, `there is an expectation
of give-and-take between the partners, negotiating differences that
were otherwise litigated.'~\cite{linder:1999}.  The previous section
explains how rather than shared responsibility, this partnership is
characterised by disputed responsibility. Sharing knowledge, however,
is certainly regarded by both partners as integral to this
relationship and building trust and collaboration is a dominant theme
running through not only the strategy documents but also the responses
from the private sector.

\subsection{The practice of information sharing as a partnership}

There can be little doubt that the main form of cooperation within the
public-private partnership is found in the shared emphasis on
information sharing~\cite{dunncavelty+suter:2009}. In July 2010, the
US Government Accountability Office published a report entitled
{\emph{Critical Infrastructure Protection: Key Private and Public
Cyber Expectations Need to Be Consistently
Addressed}}~\cite{usgao:2010}. The purpose of the study was to clarify
the partnership expectations of both the public and private sectors
and to determine the extent to which those expectations were being
met. The study was limited to five key critical infrastructure sectors
deemed to be most reliant on cyber security: communications, defence
industrial base, energy, banking and finance, and information
technology.

The provision of timely and actionable cyber threat and alert
information emerges as a key expectation of the partnership from both
the public and the private sectors but there are a number of obstacles
to sharing information from both perspectives.  The private sector
reports that it is not always easy to immediately distinguish between
some kind of technical problem, a low level attack and a large scale
sustainable attack.  In addition, it sometimes runs counter to their
commercial interests to report vulnerabilities. Finally, for private
security firms, sharing information with the government about attacks,
could lead to it being shared with their competitors. Their business
model is reliant on obtaining, holding and selling information, not
sharing it~\cite{usgao:2010}.

The public sector also encounters limitations to sharing
information. Classified contextual information cannot be shared with
individuals who do not have adequate security clearances. Even those
working in the private sector who do have security clearance can often
do nothing with classified information because to take action on it
would expose it. In addition, there is a high expectation that threat
information shared from the public to the private sector will be
accurate and this leads to extensive and stringent review and revision
processes that also delay the release of time critical information
~\cite{usgao:2010}.  This problem of sharing information has
persistently been regarded as a key impediment to cyber security and
in testimony before a US Congressional hearing on cyber security in 2011,
a senior official highlighted this as one of two main areas that
needed improvement~\cite{wilshusen:2011}.

\subsection{Key objectives and markers of success}

By the late 1990s, the critical literature looking at public-private
partnerships was maturing and there was a realisation that evaluating
these arrangements was complex and under-researched. Essentially,
there was little evidence to suggest what the success/failure rate of
these arrangements was. In fact, there was not really even a
conceptual framework for doing so. In 1999, {\emph{American Behavioural
Scientist}} published a special issue dedicated to these questions. In
the introduction, Rosenau summarises~\cite{rosenau:1999} many of the
journal arguments when she writes that `in general, partnering success
is more likely if (a) key decisions are made at the very beginning of
a project and set out in a concrete plan, (b) clear lines of
responsibility are indicated, (c) achievable goals are set down, (d)
incentives for partners are established, and (e) progress is
monitored.' She also identifies a set of criteria for the measurement
of success -- some of which are useful in considering this case,
particularly accountability and possible conflicts of interest.   

In terms of conflict of interest, she makes the case that partnerships
do not (as many assume) necessarily reduce regulation. If the
interests of the private sector are misaligned with normative goals
like care for the vulnerable (for example, old age homes) then the
government must monitor and regulate to ensure the profit motive does
not supersede the intended delivery of service~\cite{rosenau:1999}.
Here we see the profile of one of the central problems of this
public-private partnership; the expectation that the private sector
will invest in cyber security beyond their cost/benefit analysis to
fully accommodate the public interest -- in other words, to ensure
national security. Because market incentives are not adequate to
promote this level of security, oversight and some level of regulation
are necessary. A 2013 US Government Accountability Office
report~\cite{usgao:2013} found that many of the experts they consulted
argued that the private sector had not done enough to protect critical
infrastructure against cyber threats. The private sector explanation
for not fully engaging in the government’s cyber security strategy was
that the government had failed to make a convincing business case that
mitigating threats warranted substantial new investment.  Dunn Cavelty
and Suter argue that while public private cooperation is necessary,
the way it is organised and conceptualised needs to be rethought. They
propose to do so through governance theory and they find that `CIP
policy should be based as far as possible on self-regulating and
self-organising networks'. By this, they mean that `...the
government’s role no longer consists of close supervision and
immediate control, but of coordinating networks and selecting
instruments that can be used to motivate these networks for CIP
tasks.'~\cite{dunncavelty+suter:2009}. This may provide some forward
momentum though Rosenau makes the point here that a
public-private partnership cannot be regarded as a success if it
`results in lower quality of public policy services, the need for more
government oversight, and the need for expensive monitoring, even if
it appears to reduce costs'. Perhaps more problematically for Dunn
Cavelty and Suter’s recommendation is the problem of accountability.

On accountability, Rosenau writes that because these partnerships
often see policy decisions and practices that are normally reserved
for elected officials delegated to the private sector, accountability
is essential to maintaining a healthy democratic order. If
responsibility and accountability can be devolved to private actors,
the central principle that political leaders and governments are held
to account is undermined~\cite{rosenau:1999}.  For many scholars, to
ensure effective accountability in a public-private partnership, the
specifics of roles and responsibilities must be made clear at the
outset and goals must be clearly articulated.  In addition, Stiglitz
and Wallsten~\cite{stiglitz+wallsten:1999} observe that in doing so,
it becomes clear when additional incentives and resources are
necessary to achieve agreed goals and these must be provided if
accountability is to be sustained.  In cases such as cyber security,
in which the public good is the end goal for government, as with the
alignment of interests discussed above, accountability does not appear
to emerge from market forces alone.  This is not to suggest that
public-private partnerships cannot be successful when interests and
objectives diverge, but in the view of Stiglitz and Wallsten, in these
cases `more attention needs to be placed on the
incentive-accountability structure'~\cite{stiglitz+wallsten:1999}.

The 2010 US GAO report~\cite{usgao:2010} referred to previously is
also useful for the analysis of key objectives of this partnership and
for measuring its success.  The report found that in addition to
information sharing, there were two main expectations that the
government holds of the private sector in this partnership. First, it
was expected that they would commit to execute plans and
recommendations such as best practices. This is important because it
is an example of the government shifting responsibility to the private
sector in the understanding that if the private sector responds, then
regulation can be avoided. The study reported that four of the five
sectors examined were meeting government expectations to a
`great/moderate' degree. The exception was the IT sector which was
reported as demonstrating ‘little/no’ commitment to execute plans and
recommendations such as best practice.  In fact, the IT sector meets
only one out of ten services expected by the government to a
`great/moderate' degree – technical expertise. On all other criteria,
this sector ranked at `some' or `little/no'~\cite{usgao:2010}. Given
the reliance of the other sectors on the IT sector, this deficit is
particularly concerning and to some degree, has to undermine the
others' compliance.

The second key expectation (apart from information sharing) identified
in the GAO report is that the private sector will provide appropriate
staff and resources. Only banking/finance and commerce were reported
to be meeting this expectation to a `great/moderate' degree with
defence industrial base, energy and IT all being ranked at `some'. 

\section{Conclusions}

At this stage, prior to the fieldwork interviews, it is possible to
draw preliminary conclusions. First, and somewhat surprisingly given
its centrality in successive cyber security policies, exactly what
this `partnership' entails has always been unclear.  Unpacking it has
revealed that there are inherent tensions and misaligned objectives
that are not in keeping with expectations of public-private
partnership arrangements. The partnership is consistently referred to
in strategy documents using normative, value based language rather
than clear statements outlining legal authority, responsibility and
rights. Although politicians subscribe to the notion that there exists
(or should exist) a deeply entrenched norm of cooperation between the
government and private sector this appears not to be the case. Rather,
the private sector has consistently expressed an aversion to accepting
responsibility for national security and regard cyber security within
a cost/benefit framework rather than a `public good' framework. 

The second conclusion to arise from this study is that we are
witnessing a unique approach to `out-sourcing' national security that
has implications for conceptions of state power, global security and
international partnerships and resource-sharing. States with greater
government control over critical infrastructure and also over their
information infrastructure potentially have a significant advantage in
that they are able to control and shape their response to cyber
insecurity with greater autonomy and agency. This is of particular
relevance to emerging UK cyber security strategy and needs to be
considered more thoroughly from a research, policymaking and national
infrastructure perspective.

\bibliographystyle{unsrt}
\bibliography{CSSS2015}

\end{document}
